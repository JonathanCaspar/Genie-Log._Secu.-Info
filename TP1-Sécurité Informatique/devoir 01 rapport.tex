\documentclass{article}
\usepackage[utf8]{inputenc}

\title{Devoir \# 1\linebreak S\'{e}curit\'{e} informatique - IFT $3275 /$ IFT 6271}
\author{Jonathan Caspar (20059041) - Johnny Pho (20046014)}
\date{28 Février 2019}

\begin{document}

\maketitle
{\Large\bfseries Partie Théorique\par}

\section{\normalsize Soit un masque jetable utilisant une clef $k = 0^l$
(composée
seulement de zéros). Nous remarquons que $k \oplus m = m$ et que notre
message chiffré est en fait notre message clair! De ce fait, est-il nécessaire
d’utiliser des générateurs de bits qui produisent seulement des clefs $k\neq 0^{l}$
pour utiliser un masque jetable?}

//TODO

\section{\normalsize Soit un r\'{e}seau de Feistel compos\'{e} de deux ''rounds'' utilisant les fonctions de ``rounds'' $f_{1}$ et $f_{2}$. D\'{e}montrez que :\linebreak Feistel $f_{1}, f_{2}(L_{0},\ R_{0})=(L_{2},\ R_{2}) \Rightarrow$ Feistel $f_{2}, f_{1}(R_{2},\ L_{2})=(R_{0},\ L_{0})$}

//TODO

\section{\normalsize  D\'{e}montrez la propri\'{e}t\'{e} de compl\'{e}mentarit\'{e} de DES, c'est- \`{a}-dire que :
$$
DES_{k}(m)=\overline{DES_{\overline{k}}(\overline{m})}
$$
pour toute clef $k$ et message $m$ (o\`{u} $\overline{x}$ repr\'{e}sente la n\'{e}gation logique bit \`{a} bit de x).}

//TODO

\section{\normalsize}

//TODO
\end{document}
